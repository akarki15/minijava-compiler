\documentclass[11pt]{article}

\oddsidemargin 0in
\textwidth 6.5in
\topmargin -0.5in
\textheight 8.75in

\begin{document}
\begin{center}
\Large \bf Compiler Project, Stage 5: Instruction Selection!\\ \mbox{} \\
\large Computer Science 371 \\
\large Amherst College \\
\large Spring 2014
\end{center}

This assignment is due on {\bf Friday, May 9}.

\section{Getting Ready}

Go into your \verb'cs371' directory and issue the following commands:
\begin{verbatim}
      cp -r hw4 hw5          
      cp -r ~lamcgeoch/cs371/hw5/* hw5
      cd hw5
      rm Makefile
      ln -s Makefile5 Makefile
\end{verbatim}
In addition, issue the command \verb'chmod -R a+w hw5' if you are working in a group directory.
The effect of these commands is to give you a new directory, \verb'hw5'.  You have many new directories and files.

Try running \verb'make'.  Assuming that your \verb'hw4' files compile correctly, it should run without errors.

\section{New files in this Release}

\mbox{}\par\noindent{\bf Makefile5}: You should save your old Makefile and then rename this one to replace it.

\mbox{}\par\noindent{\bf arch}:  Contains object code required for library calls (for printing, string manipulation, etc.) on different architectures.

\mbox{}\par\noindent{\bf tests5}:  (Note the name!)  Contains test programs and scripts for running them.  You should strive to have your compiler work correctly on all the .java files in this directory. 

\mbox{}\par\noindent{\bf minijava/Canonical.java}:  A main program for converting .icode1 files into linearized .icode2 files.

\mbox{}\par\noindent{\bf minijava/Interp2.java}:  A main program for interpreting .icode2 files.

\mbox{}\par\noindent{\bf minijava/CodeGen.java}:  A main program for converting .icode2 files into assembly code.

\mbox{}\par\noindent{\bf minijava/Canon}:  Code for canonicalization.

\mbox{}\par\noindent{\bf minijava/BackEnd}:  Lots of code for various back-end activities.

\mbox{}\par\noindent{\bf minijava/BackEnd/Arch/Linux64/CodeGen.java}:  {\bf This is the file that you should fill in to do code generation.}

\section{Your Task}
Your task is to do instruction selection, working in the file listed above.  We'll talk lots about this task in class, and I may have further written instructions and/or advice.  

One way to print the result of your instruction selection is to uncomment lines that you will find in {\bf minijava/BackEnd/ICode/Method.java}.

\section{Running Various Scripts}

To compile, link, and execute a .java program, you can type:

\begin{verbatim}
      ./doit Hanoi.java
\end{verbatim}

Once you have an executable, you can simply run:
\begin{verbatim}
      ./Hanoi
\end{verbatim}

You can run separate programs for various phases of compilation.  To produce intermediate code for the Linux64 architecture, do:
\begin{verbatim}
      ./compile Hanoi.java
\end{verbatim}

To linearize the resulting intermediate code, do:
\begin{verbatim}
      ./canonical Hanoi.icode1
\end{verbatim}

(You can run an interpreter on .icode2 files, but you probably won't need to do this.  The proper sequence of commands would be:
\begin{verbatim}
      ./compile -target simple Hanoi.java
      ./canonical Hanoi.icode1
      ./interp2 Hanoi.icode2
\end{verbatim}
In other words, the interpreter only works when you are targetting the simple architecture.)

To create assembly code from .icode2 files, do:
\begin{verbatim}
      ./codegen Hanoi.icode2
\end{verbatim}

To assemble and link the result assembly file, do:
\begin{verbatim}
      ./link Hanoi.s
\end{verbatim}

\section{Submitting Your Work}
To do an electronic submission in the usual way.

\end{document}

